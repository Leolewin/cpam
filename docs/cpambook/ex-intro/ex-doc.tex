\begin{Exercise}[title={Documentation},difficulty=1]
\label{ex:doc}
\Question
Go's documentation can be read with the \prog{go doc} program, which is
included the Go distribution.

\prog{go doc hash} gives information about the \package{hash} package. Reading the
documentation on \package{compress} gives the following result:
\vskip\baselineskip
\begin{display}
\pr \user{go doc compress}
SUBDIRECTORIES

        bzip2
        flate
        gzip
        lzw
        testdata
        zlib
\end{display}
\vskip\baselineskip
With which \prog{go doc} command can you read the documentation of \package{gzip} contained in
\package{compress}?

\end{Exercise}

\begin{Answer}
\Question
The package \package{gzip} is in a \emph{subdirectory} of
\package{compress}, so you will only need\quad \texttt{go doc compress/gzip}.

Specific functions inside the ``Go manual'' can also be accessed. For
instance the function \func{Printf} is described in \package{fmt}, but to
only view the documentation concerning this function use: \prog{go doc fmt Printf}{} .

You can even display the source code with: \prog{go doc -src fmt Printf} .     

All the built-in functions are also accesible by using go doc: \prog{go doc builtin}.
\end{Answer}
