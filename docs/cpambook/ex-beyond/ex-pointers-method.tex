\begin{Exercise}[title={Method calls},difficulty=8]
\label{ex:methodcalls}
\Question \label{ex:methodcalls q1} Suppose we have the following
program: TODO: aint true anymore in Go1 (vector is gone)
\begin{lstlisting}
package main

import "container/vector"

func main() {
	k1 := vector.IntVector{}
	k2 := &vector.IntVector{}
	k3 := new(vector.IntVector)
	k1.Push(2)
	k2.Push(3)
	k3.Push(4)
}
\end{lstlisting}
What are the types of \var{k1}, \var{k2} and \var{k3}?

\Question Now, this program compiles and runs OK. All the \func{Push}
operations work even though the variables are of a different type. The
documentation for \func{Push} says:
\begin{quote}
func (p *IntVector) Push(x int)
Push appends x to the end of the vector.
\end{quote}
So the receiver has to be of type \type{*IntVector}, why does the code
above work then?

\end{Exercise}

\begin{Answer}
\Question The type of \var{k1} is \type{vector.IntVector}. Why? We use 
a composite literal (the \verb|{}|), so we get a value of that type
back. The variable \var{k2} is of \type{*vector.IntVector}, because we
take the address (\verb|&|) of the composite literal. And finally
\var{k3} has also the type \type{*vector.IntVector}, because \func{new}
returns a pointer to the type.

\Question The answer is given in \cite{go_spec} in the section ``Calls'',
where among other things it says:
\begin{quote}
A method call \func{x.m()} is valid if the method set of (the type of)
\var{x}
contains \func{m} and the argument list can be assigned to the parameter list
of \func{m}. If \var{x} is addressable and \var{\&x}'s method set
contains \func{m}, \func{x.m()} is shorthand for \func{(\&x).m()}.
\end{quote}
In other words because \var{k1} is addressable and
\type{*vector.IntVector} \emph{does} have the \func{Push} method, the
call \lstinline{k1.Push(2)} is translated by Go into 
\lstinline{(&k1).Push(2)} which makes the type system happy again (and
you too --- now you know this).\footnote{Also see section
``\titleref{sec:methods}'' in this chapter.}

\end{Answer}
